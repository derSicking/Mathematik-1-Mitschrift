\documentclass[a4paper,12pt,oneside,openany]{book}

\usepackage[ngerman]{babel}
\usepackage{blindtext}
\usepackage{anyfontsize}
\usepackage[dvipsnames]{xcolor}
\usepackage[no-math]{fontspec}
\usepackage{graphicx}
\usepackage{setspace}
\usepackage{titlesec}
\usepackage{calc}
\usepackage{fancyhdr}
\usepackage{titletoc,tocloft}
\usepackage{lastpage}
\usepackage{ifthen}
\usepackage{tikz}
\usepackage{thmbox}
\usepackage{amssymb}
\usepackage{enumitem}
\usepackage{amsfonts}
\usepackage{scrextend}
\usepackage{array}
\usepackage{stackengine}
\usepackage{multirow}
\usepackage{csquotes}
\usepackage{contour}
\usepackage{ulem}
\usepackage{amsmath}
\usepackage{parskip}
\usepackage{marvosym}
\usepackage[abspath]{currfile}

\usepackage[a4paper, margin=1in]{geometry}
\usepackage{pgfplots}

\pgfplotsset{width=9cm, compat=1.18}
% \usepgfplotslibrary{external}
% \tikzexternalize

\newcommand{\pipe}{\;|\;}

\renewcommand{\baselinestretch}{1.5}

\setcounter{secnumdepth}{3}

\setlength{\parskip}{5pt}

\titleformat{\chapter}[hang]{\normalfont\Huge\bfseries}{\thechapter}{1cm}{}[\titlerule]
\titlespacing*{\chapter}{0cm}{0cm}{0.5cm}

\titleformat{\section}[hang]{\normalfont\Large\bfseries}{\thesection}{0.5cm}{}[]
\titlespacing*{\section}{0cm}{0.5cm}{0.3cm}

\titleformat{\subsection}[hang]{\normalfont\large\bfseries}{\thesubsection}{0.5cm}{}[]
\titlespacing*{\subsection}{0cm}{0.5cm}{0.3cm}

\titleformat{\subsubsection}[hang]{\normalfont\large\bfseries}{\thesubsubsection}{0.5cm}{}[]
\titlespacing*{\subsubsection}{0cm}{0.5cm}{0.2cm}

\newcommand{\pskip}{\vspace{12pt}}
\newcommand{\lskip}{\vspace{2mm}}
\newcommand{\textspaced}[1]{\hspace{3mm}#1\hspace{3mm}}

\pagestyle{fancy}
\fancypagestyle{plain}{}
\fancyhf{}
\renewcommand{\headrulewidth}{0pt}
\fancyhead{}
\fancyfoot[R]{\thepage}

\graphicspath{ {./images/} }

\setlength{\parindent}{0pt}

\setsansfont{LINETO-BROWN-PRO-LIGHT.TTF}[
	Path = \currfileabsdir fonts/,
	BoldFont = LINETO-BROWN-PRO-REGULAR.TTF ,
	ItalicFont = LINETO-BROWN-PRO-LIGHT.TTF ,
	BoldItalicFont = LINETO-BROWN-PRO-REGULAR.TTF 
]

\renewcommand\familydefault{\sfdefault}

\definecolor{fhblue}{RGB}{0, 20, 160}

\setlist[enumerate]{label=\arabic*), itemsep=-2pt}
\setlist[itemize]{itemsep=-2pt}

\renewcommand{\ULdepth}{2pt}
\contourlength{1pt}

\newcommand{\myul}[1]{
  \uline{\phantom{#1}}\llap{\contour{white}{#1}}
}

\newcounter{definition}[chapter]

\newenvironment{leftrule}{\normalfont\normalsize\begin{addmargin}[-3.5mm]{0cm}\begin{leftbar}\vspace{1.5mm}\begin{addmargin}[4mm]{0cm}}{\end{addmargin}\end{leftbar}\end{addmargin}}
\newenvironment{definition}[1][]{\refstepcounter{definition}\vspace{5mm}\bfseries\large Definition \thechapter.\thedefinition \ifthenelse{\equal{#1}{}}{}{: #1}\begin{leftrule}}{\end{leftrule}}
\newenvironment{satz}[1][]{\refstepcounter{definition}\vspace{5mm}\bfseries\large Satz \thechapter.\thedefinition\ifthenelse{\equal{#1}{}}{}{: #1}\begin{leftrule}}{\end{leftrule}}
\newenvironment{benanntersatz}[1][]{\refstepcounter{definition}\vspace{5mm}\bfseries\large\thechapter.\thedefinition #1\begin{leftrule}}{\end{leftrule}}
\newenvironment{beispiele}[1][]{\vspace{2mm}\normalfont\normalsize \textbf{Beispiele}\ifthenelse{\equal{#1}{}}{:}{: #1}\begin{enumerate}[label=\arabic*)]}{\end{enumerate}}
\newenvironment{beispielenolist}[1][]{\vspace{2mm}\normalfont\normalsize \textbf{Beispiele}\ifthenelse{\equal{#1}{}}{:}{: #1}\newline\vspace{-8mm}\begin{addmargin}[5mm]{0cm}}{\end{addmargin}}
\newenvironment{beispiel}[1][]{\vspace{2mm}\normalfont\normalsize \textbf{Beispiel}\ifthenelse{\equal{#1}{}}{:}{: #1}\newline\vspace{-8mm}\begin{addmargin}[5mm]{0cm}}{\end{addmargin}}
\newenvironment{bemerkung}[1][]{\vspace{2mm}\normalfont\normalsize \textbf{Bemerkung}\ifthenelse{\equal{#1}{}}{:}{: #1}\newline\vspace{-8mm}\begin{addmargin}[5mm]{0cm}}{\end{addmargin}}
\newenvironment{bemerkungen}[1][]{\vspace{2mm}\normalfont\normalsize \textbf{Bemerkungen}\ifthenelse{\equal{#1}{}}{:}{: #1}\begin{enumerate}[label=\arabic*)]}{\end{enumerate}}
\newenvironment{bemerkungennolist}[1][]{\vspace{2mm}\normalfont\normalsize \textbf{Bemerkungen}\ifthenelse{\equal{#1}{}}{:}{: #1}\newline\vspace{-8mm}\begin{addmargin}[5mm]{0cm}}{\end{addmargin}}
\newenvironment{bemerkungenbsp}[1][]{\vspace{2mm}\normalfont\normalsize \textbf{Bemerkungen und Beispiele}\ifthenelse{\equal{#1}{}}{:}{: #1}\begin{enumerate}[label=\arabic*)]}{\end{enumerate}}
\newenvironment{bemerkungenbspnolist}[1][]{\vspace{2mm}\normalfont\normalsize \textbf{Bemerkungen und Beispiele}\ifthenelse{\equal{#1}{}}{:}{: #1}\newline\vspace{-8mm}\begin{addmargin}[5mm]{0cm}}{\end{addmargin}}

\newcommand{\begriff}[1]{\myul{#1}}

% \newcommand{\überschrift}[1]{\vspace{5mm}\large\textbf{#1}\vspace{2mm}\normalsize}
\newcommand{\überschrift}[1]{\subsubsection{#1}}

\author{Frederik Sicking}
\title{Mathematik 1 Mitschrift}

\begin{document}

\begin{titlepage}

	\includegraphics[height=0.45in]{logo}
	\hspace{\fill}
	\includegraphics[height=0.45in]{ETI-logo}

	\vspace*{5cm}
	\fontsize{36}{36}\selectfont \textbf{Mathematik 1}\\
	\fontsize{28}{28}\selectfont \textcolor{fhblue}{\textbf{Mitschrift}}

	\vspace{1.5cm}
	\fontsize{20}{20}\selectfont Frederik Sicking

	\vspace{\fill}
	\begin{flushright}

		\fontsize{12}{15}\selectfont
		Modul: \textbf{Mathematik 1} \\
		\textbf{Prof. Dr. Gernot Bauer}

		\vspace{0.5cm}
		Wintersemester 2022 / 2023 \\
		Stand: Freitag, 11.11.2022

		\vspace{1.5cm}
		\textbf{Frederik Sicking}\\
		\vspace{0.5cm}
		frederik.sicking@fh-muenster.de

	\end{flushright}
\end{titlepage}

\frontmatter
\pagenumbering{Roman}

% \chapter*{Vorwort}
% \setcounter{page}{1}
% \blindtext[2]
% \newpage

\setcounter{tocdepth}{3}
\setlength{\cftsecindent}{5mm}
\setlength{\cftsubsecindent}{1cm}
\setlength{\cftsubsubsecindent}{1.5cm}
\tableofcontents

\mainmatter
\chapter{Grundlagen}
\setcounter{page}{1}
\section{Aussagelogik}
Im Unterschied zur Umgangssprache benutzt die Mathematik eine sehr
präzise Sprechweise, die wir hier einführen wollen.

\subsection{Aussagen}
Sachverhalte der Realität werden in Form von Aussagen erfasst.

\begin{definition}[Aussage]
	Unter einer \begriff{Aussage} versteht man ein sinnvolles sprachliches Gebilde,
	das entweder wahr oder falsch sein kann.
\end{definition}

\begin{beispielenolist}
	\vspace*{-5mm}
	\begin{tabular}{p{0.05\textwidth}m{0.7\textwidth}>{\centering\arraybackslash}p{0.2\textwidth}}
		   &                                           & ist Aussage \\
		1) & 5 ist kleiner als 3.                      & ja          \\
		2) & Kiew ist die Hauptstadt der Ukraine.      & ja          \\
		3) & Das Studium der Mathematik ist schwierig. & ja          \\
		4) & Nach dem Essen Zähne putzen!              & nein        \\
		5) & Nachts ist es kälter als draußen.         & nein
	\end{tabular}
\end{beispielenolist}

\newcommand{\wahr}{\textrm{wahr}}
\newcommand{\falsch}{\textrm{falsch}}
\pskip
Die Werte \begriff{\wahr} und \begriff{\falsch} heißen Warheitswerte. Jede Aussage hat genau
einen dieser beiden Warheitswerte. Das heißt aber nicht, dass der Warheitswert
auch bekannt ist.

\newpage
\begin{beispielenolist}[Fortsetzung]
	\vspace*{-5mm}
	\begin{tabular}{p{0.05\textwidth}p{0.7\textwidth}>{\centering\arraybackslash}p{0.2\textwidth}}
		   &                                                                                    & ist Aussage \\
		6) & Der Sommer 2023 wird erneut der heißeste in Europa seit Beginn der Aufzeichnungen. & ja          \\
		7) & Jede gerade Zahl größer 2 ist Summe zweier Primzahlen. (Goldbachsche Vermutung)    & ja
	\end{tabular}
\end{beispielenolist}

\begin{bemerkung}
	Eine Aussage, die einen mathematischen Sachverhalt beschreibt und \wahr ~ist, wird als \begriff{Satz} bezeichnet.
\end{bemerkung}

\subsection{Verknüpfung von Aussagen}
Im folgenden Stehen lateinische Großbuchstaben \( A, B, C, \dots \) als Platzhalter (Variablen) für Aussagen.

\überschrift{Die \textquote{und}-Verknüpfung (Konjugation)}

Eine zusammengesetzte Aussage der Form
\begin{align*}
	A \textrm{~und~} B & \hspace*{1.5cm}\textsf{(Kurzbezeichnung:~} A \land B \textsf{)}
\end{align*}
ist \wahr, wenn beide Aussagen \wahr~sind. Andernfalls ist sie \falsch.

\newcommand{\w}{\textrm{w}}
\newcommand{\f}{\textrm{f}}

Der Warheitswert der zusammengesetzten Aussage in Abhängigkeit von \( A \) und \( B \) kann durch folgende
\begriff{Verknüpfungstabelle} (oder \begriff{Wahrheitstafel}) ausgedrückt werden. (\w~für \wahr, \f~für \falsch)

\begin{table}[h]
	\centering
	\begin{tabular}{c | c | c}
		\( A \) & \( B \) & \( A \land B \) \\
		\hline
		\w      & \w      & \w              \\
		\w      & \f      & \f              \\
		\f      & \w      & \f              \\
		\f      & \f      & \f
	\end{tabular}
\end{table}

\newpage
\begin{beispiel}[\textquote{und}]
	\vspace{-1.5cm}
	\begin{align*}
		A & : \textsf{7 ist ungerade.}                                        & (\wahr)   \\
		B & : \textsf{17 < 4}                                                 & (\falsch) \\
		C & : \textsf{Für alle reellen Zahlen~} x \textsf{~gilt:~} x^2 \geq 0 & (\wahr)
	\end{align*}
	Die Aussage \textquote{7 ist ungerade und 17 < 4} (\( A \land B \)) ist \falsch.
	Die Aussage \( A \land C \) ist \wahr.
\end{beispiel}

\überschrift{Die \textquote{oder}-Verknüpfung (Alternative, Disjunktion)}

Eine zusammengesetzte Aussage der Form
\begin{align*}
	A \textrm{~oder~} B & \hspace*{1.5cm}\textsf{(Kurzbezeichnung:~} A \lor B \textsf{)}
\end{align*}
ist \wahr, wenn mindestens  eine der beiden Aussagen \wahr~ist. Sind beide
Aussagen \falsch, dann ist auch die zusammengesetzte Aussage \( A \lor B\) \falsch. Wahrheitstafel:

\begin{table}[h]
	\centering
	\begin{tabular}{c | c | c}
		\( A \) & \( B \) & \( A \lor B \) \\
		\hline
		\w      & \w      & \w             \\
		\w      & \f      & \w             \\
		\f      & \w      & \w             \\
		\f      & \f      & \f
	\end{tabular}
\end{table}

\begin{beispiel}[\textquote{oder}]
	\vspace{-1.5cm}
	\begin{align*}
		A & : \textsf{Allerheiligen ist am 1.11.} & (\wahr)                                       \\
		B & : \textsf{Die Erde ist eine Scheibe.} & (\falsch)                                     \\
		C & : \textsf{Heute ist Montag.}          & (\wahr / \falsch \textsf{~je nach Wochentag})
	\end{align*}
	Die Aussage \textquote{Allerheiligen ist am 1.11. oder die Erde ist eine Scheibe} (\( A \lor B \)) ist \wahr,
	die Aussage \( A \lor C \) ist ebenfalls immer \wahr. \( B \lor C \) ist dagegen nur an einem Montag \wahr, sonst \falsch.
\end{beispiel}

\begin{bemerkung}
	Im Alltagssprachgebrauch trifft man häufig auf die Verknüpfung von Aussagen mit
	\textquote{und/oder}, etwa \textquote{Ich komme heute und/oder morgen}.
	Mathematisch ist das nicht sinnvoll, ein einfaches \textquote{\textrm{oder}} drückt den Sachverhalt
	bereits treffend aus.
\end{bemerkung}

\newpage
\überschrift{Die Negation (\textquote{nicht})}

Eine Aussage der Form
\begin{align*}
	\textrm{nicht~} A & \hspace*{1.5cm}\textsf{(Kurzbezeichnung:~} \lnot A \textsf{)}
\end{align*}
ist \wahr, wenn \( A \) \falsch~ist. Sie ist \falsch, wenn \( A \) \wahr~ist.
Die Aussage \( \lnot A \) heißt die \begriff{Negation} von \( A \). Wahrheitstafel:

\begin{table}[h]
	\centering
	\begin{tabular}{c | c }
		\( A \) & \( \lnot A \) \\
		\hline
		\w      & \f            \\
		\f      & \w
	\end{tabular}
\end{table}

Die Negation (oder Verneinung) kehrt den Warheitswert einer Aussage um.

\überschrift{Die \textquote{wenn-dann}-Verknüpfung (Implikation, Schlussfolgerung)}

Eine zusammengesetzte Aussage der Form
\begin{align*}
	A \textrm{~impliziert~} B & \hspace*{1.5cm}\textsf{(Kurzbezeichnung:~} A \implies B \textsf{)}
\end{align*}
ist \falsch, falls \( A \) \wahr~und \( B \) \falsch~ist.
Andernfalls ist sie \wahr. Wahrheitstafel:

\begin{table}[h]
	\centering
	\begin{tabular}{c | c | c}
		\( A \) & \( B \) & \( A \implies B \) \\
		\hline
		\w      & \w      & \w                 \\
		\w      & \f      & \f                 \\
		\f      & \w      & \w                 \\
		\f      & \f      & \w
	\end{tabular}
\end{table}

Weitere übliche Sprechweisen für \(A \implies B\) sind:
\begin{itemize}
	\item Wenn \(A\) gilt, dann gilt (auch) \(B\).
	\item Aus \(A\) folgt \(B\).
	\item Aus \(A\) kann man \(B\) schlussfolgern. \newpage
	\item \(A\) ist hinreichend für \(B\).
	\item \(A\) ist eine hinreichende Bedingung für \(B\).
	\item \(B\) ist notwendig für \(A\).
	\item \(B\) ist eine notwendige Bedingung für \(A\).
\end{itemize}

\begin{beispiele}
	\item\mbox{}\vspace*{-1cm}\begin{align*}
		A & : \textsf{Weihnachten ist am 25.12.}          & (\wahr)   \\
		B & : \textsf{Ich fresse einen Besen.}            & (\falsch) \\
		C & : \textsf{Weihnachten fällt auf Ostern.}      & (\falsch) \\
		D & : \textsf{Die Goldbachsche Vermutung stimmt.} & (?)
	\end{align*}
	Die Aussage \textquote{Wenn Weihnachten am 25.12. ist, dann fresse ich einen Besen}
	(\(A \implies B\)) ist \falsch.

	Dagegen ist die Aussage \textquote{Wenn Weihnachten auf Ostern fällt, dann fresse ich einen Besen}
	(\(C \implies D\)) \wahr!

	Die Aussage \(D \implies A\) ist \wahr, unabhängig davon, ob Goldbach recht hatte. (\(D\) ändert
	nichts daran, dass Weihnachten am 25.12. ist.)

	\item\mbox{}\vspace*{-1cm}\begin{align*}
		A & : \textsf{Es regnet.}           \\
		B & : \textsf{Die Straße ist nass.}
	\end{align*}
	Die Implikation \(A \implies B\) ist \wahr. Wenn es regnet, dann ist die Straße nass. Also
	ist \(A\) hinreichend für \(B\).

	Zugleich ist \(B\) notwendig für \(A\), denn:

	Wenn die Straße nicht nass ist, dann regnet es nicht.

	\(A\) ist allerdings nicht notwendig für \(B\): Die Straße kann auch nass werden, ohne dass
	es regnet.
\end{beispiele}

\begin{bemerkung}
	Es mag überraschen, dass Schlussfolgerungen aus falschen Voraussetzungen immer wahr sind.
	Anders ausgedrückt:

	Aus einer falschen Voraussetzung kann man jede beliebige Behauptung schlussfolgern, sei
	diese Behauptung auch wahr oder falsch.

	Zum Beispiel kann man aus der Voraussetzung \textquote{\(0 = 1\)} die Behauptung
	\textquote{Einstein ist der Papst} schlussfolgern.
\end{bemerkung}

\überschrift{Die \textquote{genau-dann-wenn}-Verknüpfung (Äquivalenz)}

Eine zusammengesetzte Aussage der Form
\newcommand{\equivalent}{\Longleftrightarrow}
\begin{align*}
	A \textrm{~ist äquivalent zu~} B & \hspace*{1.5cm}\textsf{(Kurzbezeichnung:~} A \equivalent B \textsf{)}
\end{align*}
bedeutet, dass die Warheitswerte von \(A\) und \(B\) gleich sind. Wahrheitstafel:

\begin{table}[h]
	\centering
	\begin{tabular}{c | c | c}
		\( A \) & \( B \) & \( A \equivalent B \) \\
		\hline
		\w      & \w      & \w                    \\
		\w      & \f      & \f                    \\
		\f      & \w      & \f                    \\
		\f      & \f      & \w
	\end{tabular}
\end{table}

Weitere übliche Sprechweisen für \(A \equivalent B\) sind:
\begin{itemize}
	\item \(A\) gilt genau dann, wenn \(B\) gilt.
	\item \(A\) gilt dann und nur dann, wenn \(B\) gilt.
	\item \(A\) ist gleichbedeutend mit \(B\).
	\item \(A\) impliziert \(B\) und \(B\) impliziert \(A\).
	\item \(A\) ist notwendig und hinreichend für \(B\).
	\item \(B\) ist notwendig und hinreichend für \(A\).
\end{itemize}

\begin{beispiel}
	Es sei \(m\) eine ganze Zahl.
	\begin{align*}
		A & : m \textsf{~ist durch~} 6 \textsf{~teilbar.}                        \\
		B & : m \textsf{~ist durch~} 2 \textsf{~und durch~} 3 \textsf{~teilbar.}
	\end{align*}
	Es gilt \(A \equivalent B\).
\end{beispiel}

\überschrift{Komplizierte Verknüpfungen}

Aus gegebenen Aussagen können durch Zusammensetzen schrittweise komplizierte Aussagen
aufgebaut werden, zum Beispiel:
\begin{gather*}
	A \implies (B \lor C) \\
	\lnot(A \equivalent (B \lor (\lnot C)))
\end{gather*}
Ähnlich wie in der Zahlenalgebra (zum Beispiel \textquote{Punkt vor Strich}) gibt es auch
in der Aussagelogik Klammerkonventionen und eine Rangfolge der Verknüpfungssymbole:

\(\lnot\) bindet stärker als \(\land\), \(\lor\),

\(\land\), \(\lor\) binden stärker als \(\implies\), \(\equivalent\).

Damit gilt zum Beispiel:
\begin{align*}
	((\lnot A) \land B) & \equivalent (\lnot(C \lor D)) \\
	                    & \;\;\Updownarrow              \\
	\lnot A \land B     & \equivalent \lnot (C \lor D)
\end{align*}

\subsection{Beweistechniken}

In Beweisen wird stets aus der Gültigkeit einer Bestimmten Aussage, der \begriff{Voraus-}
\begriff{setzung} oder \begriff{Prämisse}, auf die Gültigkeit einer anderen Aussage, der \begriff{Folgerung} oder
\begriff{Konklusion} geschlossen. Bei einem Beweis wird also gezeigt, dass eine Aussage der Form
\(A \implies B\) \wahr ~ist, wobei vorab bekannt ist, dass die Prämisse \(A\) \wahr ~ist.

Nach der Abtrennungsregel (Modus ponens, vgl. Übung 7 h) )
\[
	((A \implies B) \land A) \implies B
\]
ist damit bewiesen, dass die Aussage \(B\), die Konklusion, wahr ist. Diese Art des logischen
Schließens heißt \begriff{Deduktion}.

Bei einer deduktiven Beweisführung sollte immer möglichst klar sein, was zu beweisen ist (die
Konklusion \(B\)), was vorausgesetzt wird (die Prämisse \(A\)) und wie der Übergang \(A \implies B\),
die \begriff{Argumentation}, verläuft. Argumente werden im Deutschen häufig mit Wörtern wie
\textquote{also}, \textquote{demnach}, \textquote{folglich} oder \textquote{somit} eingeleitet.

\newcommand{\qed}{
	\vspace*{-8mm}
	\begin{flushright}
		$\blacksquare$
	\end{flushright}
}
Das Ende eines Beweises wird häufig mit dem Zeichen \textquote{$\blacksquare$} gekennzeichnet.

\überschrift{Der direkte Beweis}

Ein \begriff{direkter Beweis} liegt vor, wenn \(A \implies B\) mit Hilfe von Zwischenaussagen
\(A_1,A_2,\dots,A_n\) in Form einer Argumentationskette
\[
	A \implies A_1 \implies A_2 \implies \dots \implies A_n \implies B
\]
gezeigt wird.

\begin{beispiel}
	Zu zeigen:

	$\left. \parbox{\linewidth - 2.3cm}{
			Die Summe der Innenwinkel im regelmäßigen Fünfeck ist 540°.
		} \,\right\} B $

	Beweis:

	$\left. \parbox{\linewidth - 2.3cm}{
			Die Summe der Innenwinkel eines Dreiecks beträgt 180°.
		} \,\right\} A $

	\pskip
	$\left. \parbox{\linewidth - 2.3cm}{
			Folglich ist die Summe der Innenwinkel fünf beliebiger Dreiecke\\
			\(5 \cdot 180^\circ  = 900^\circ\).
		} \,\right\} \implies A_1 $

	\pskip
	$\left. \parbox{\linewidth - 2.3cm}{
			Also ist auch die Summe der Innenwinkel der fünf deckungsgleichen Dreiecke,
			in die sich das regelmäßige Fünfeck unterteilen lässt, 900°.
		} \,\right\} \implies A_2 $

	\begin{center}
		\vspace*{3mm}
		\begin{tikzpicture}[scale=2]
			\coordinate (M) at (0,0);
			\coordinate (A) at (0,1);

			\newcommand{\cone}{0.30901699437}
			\newcommand{\ctwo}{0.80901699437}
			\newcommand{\sone}{0.95105651629}
			\newcommand{\stwo}{0.58778525229}

			\coordinate (B) at ({\sone},{\cone});
			\coordinate (C) at ({\stwo},{-\ctwo});
			\coordinate (D) at ({-\stwo},{-\ctwo});
			\coordinate (E) at ({-\sone},{\cone});

			\draw (M) -- (A);
			\draw (M) -- (B);
			\draw (M) -- (C);
			\draw (M) -- (D);
			\draw (M) -- (E);

			\draw (A) -- (B);
			\draw (B) -- (C);
			\draw (C) -- (D);
			\draw (D) -- (E);
			\draw (E) -- (A);

			\draw (M) circle (0.2);
		\end{tikzpicture}
		\vspace*{5mm}
	\end{center}

	$\left. \parbox{\linewidth - 2.3cm}{
			Die Innenwinkel dieser fünf Dreiecke, die am Mittelpunkt des Fünfecks liegen, betragen
			zusammen 360°. Sie tragen nicht zur Summe der Innenwinkel des Fünfecks bei. Somit
			ist diese Summe \(900^\circ - 360^\circ = 540^\circ\).
		} \,\right\} \implies B $

	\vspace*{6mm}
	\qed

\end{beispiel}

\überschrift{Der indirekte Beweis (Beweis durch Wiederspruch, Reductio ad absurdum)}

Häufig ist es leichter, statt der Schlussfolgerung \(A \implies B\) die
Schlussfolgerung \(\lnot B \implies \lnot A\) zu zeigen.

Nach der Kontraposition der Implikation (vgl. Übung 7 e) ) sind beide
Schlussfolgerungen gleichbedeutend.

Man leitet also ausgehend von der Annahme \(\lnot B\) einen Wiederspruch
zu \(A\) ab.

\begin{beispiel}
	Zu zeigen:

	$\left. \parbox{\linewidth - 2.3cm}{
			Ein Dreieck, bei dem ein Innenwinkel 91° beträgt, ist
			nicht rechtwinklig.
		} \,\right\} B $

	Beweis (durch Wiederspruch):

	\pskip
	$\left. \parbox{\linewidth - 2.3cm}{
			Wir nehmen an, es gäbe ein rechtwinkliges Dreieck, bei dem
			ein Innenwinkel 91° beträgt.
		} \,\right\} \lnot B $

	\pskip
	$\left. \parbox{\linewidth - 2.3cm}{
			Folglich hat das Dreieck einen Innenwinkel, der 90° beträgt,
			und einen Innenwinkel, der 91° beträgt. Demnach ist die
			Summe der Innenwinkel des Dreiecks größer 180°.
		} \,\right\} \implies \lnot A $

	\pskip
	Das ist ein Wiederspruch dazu, dass die Summe der Innenwinkel eines
	Dreiecks 180° beträgt. Somit ist die obige Annahme (\(\lnot B\)) \falsch
	~und die ursprüngliche Behauptung (\(B\)) bewiesen.

	\qed

\end{beispiel}

\newpage
\section{Mengen, Relationen und Abbildungen}

Zu den wichtigsten Grundpfeilern der Mathematik gehört der Mengenbegriff.

\subsection{Mengenlehre}

\begin{definition}[Menge (Cantor, 1895)]
	Eine \begriff{Menge} ist eine beliebige Zusammenfassung
	von bestimmten, wohlunterschiedenen Objekten unserer Anschauung
	oder unseres Denkens zu einem Ganzen.
\end{definition}

\überschrift{Sprechweisen und Notationen}

Mengen werden mit Großbuchstaben gekennzeichnet. Die Objekte der Menge
\(M\) werden die \begriff{Elemente} von \(M\) genannt.

Ist das Objekt x ein beziehungsweise kein Element von \(M\), so schreibt man
\[x \in M \textsf{~bzw.~} x \notin M\]
Zwei Mengen \(M\) und \(N\) heißen gleich, wenn sie genau die selben
Elemente enthalten.
\[x \in M \Longleftrightarrow x \in N\]
Man schreibt dann \(M = N\). Sind \(M\) und \(N\) nicht
gleich, schreibt man \(M \neq N\).

Bei der \begriff{aufzählenden Schreibweise} zur Kennzeichnung von Mengen,
zum Beispiel
\begin{align*}
	M & = \{a,e,i,o,u\}              \\
	B & = \{-3,-2,-1,0,1,2,3,\dots\} \\
	Y & = \{-2,5\}
\end{align*}
spielt die Reihenfolge keine Rolle.

Die \begriff{beschreibende Schreibweise} hat die allgemeine Struktur
\begin{alignat*}{3}
	X                & = \{x       &  & \pipe x \textsf{~hat die Eigenschaft~} E             \} \textsf{~oder} \\
	X                & = \{x \in G &  & \pipe x \textsf{~hat die Eigenschaft~} E             \} \textsf{,}     \\
	\\
	\textsf{z.B.~} M & = \{x       &  & \pipe x \textsf{~ist Vokal im deutschen Alphabet}    \} \textsf{,}     \\
	B                & = \{x       &  & \pipe x \textsf{~ist eine ganze Zahl und~} x > -4    \} \textsf{,}     \\
	Y                & = \{x \in B &  & \pipe x \textsf{~ist Lösung von~} (x+4)(x+2)(x-5)=0  \}
\end{alignat*}

Dabei wir der senkrechte Strich (\(\pipe\)) als \textquote{mit der Eigenschaft} gelesen, und \(G\)
bezeichnet eine Grundmenge, der die Elemente x entstammen sollen.

Eine Menge, die kein Element besitzt, heißt leere Menge, und wird mit \(\emptyset\) oder \(\{ \; \}\)
bezeichnet.

In Vorgiff auf Kapitel 2 nennen wir hier einige Wichtige Zahlenmengen:
\newcommand{\R}{\mathbb{R}}
\newcommand{\N}{\mathbb{N}}
\newcommand{\Z}{\mathbb{Z}}
\newcommand{\Q}{\mathbb{Q}}
\begin{alignat*}{3}
	\R &                                                                        &  & \textsf{~die Menge der \begriff{reellen Zahlen}}     \\
	\N & = \{1,2,3,\dots \}                                                     &  & \textsf{~die Menge der \begriff{natürlichen Zahlen}} \\
	\Z & = \{\dots ,-3,-2,-1,0,1,2,3,4,\dots\}                                  &  &                                                      \\
	   & = \{x \in \R \pipe x \in \N \lor x = 0 \lor -x \in \N\}                &  & \textsf{~die Menge der \begriff{ganzen Zahlen}}      \\
	\Q & = \{x \in \R \pipe x = \frac{n}{p} \textsf{~mit~} n \in \Z, p \in \N\} &  & \textsf{~die Menge der \begriff{rationalen Zahlen}}
\end{alignat*}

\newpage
Eine Menge \(M\) heißt \begriff{Teilmenge} der Menge \(N\), in Zeichen \(M \subset N\), wenn jedes
Element von \(M\) auch Element von \(N\) ist.
\[x \in M \implies x \in N \]
Wir sagen dann auch: \(M\) ist in \(N\) enthalten, oder \(N\) ist \begriff{Obermenge} von \(M\).
\begin{center}
	\begin{tikzpicture}
		\draw[black] (0,0) circle (2) node[text=black,text width=2.5cm] {\(N\)};
		\draw[black, fill=fhblue] (0.5,0.5) circle (1) node[text=white] {\(M\)};
	\end{tikzpicture}
\end{center}
Ist \(M\) nicht Teilmenge von \(N\), so schreibt man \(M \not\subset N\).

\begin{bemerkungen}
	\item \begin{samepage}
		Mengen sind selbst wieder Objekte, das heißt sie können auch wieder zu Mengen zusammengefasst
		werden, zum Beispiel:
		\[M = \{\N,\Z,\Q,\R\}, \;\; N=\{\emptyset,1,\{1\}\}\]
		\(M\) hat 4 Elemente, \(N\) hat 3 Elemente.

		Einelementige Mengen der Form \(\{m\}\) und ihr Element \(m\) sind unterschiedliche Objekte.
	\end{samepage}

	\item \begin{samepage} Beziehungen zwischen Mengen wie zum Beispiel \(M \subset N\) lassen sich auch durch
		Verknüpfungen von Aussagen über Elementzugehörigkeiten ausdrücken:
		\[
			M \subset N \Longleftrightarrow (\underbrace{x \in M}_A \implies \underbrace{x \in N}_B) \; \circledast \\
		\]
		\(M = N\) gilt genau dann, wenn \(M \subset N\) und \(N \subset M\) denn:
		\stackMath
		\[
			\begin{array}{r @{\hspace*{3mm}} c @{\hspace*{3mm}} l}
				M \subset N \land N \subset M & \stackon{\Longleftrightarrow}{\circledast}                   & (A \implies B) \land (B \implies A)   \\
				                              & \stackon{\Longleftrightarrow}{\textsf{Tautologie}}           & (A \Longleftrightarrow B)             \\
				                              & \stackon{\Longleftrightarrow}{\textsf{Def.~} A \textsf{,} B} & (x \in M \Longleftrightarrow x \in N) \\
				                              & \stackon{\Longleftrightarrow}{\;}                            & M = N
			\end{array}
		\]
	\end{samepage}

	\item Für alle Mengen \(M\) gilt \(M \subset M\) und \(\emptyset \subset M\).
\end{bemerkungen}

\überschrift{Mengenoperationen}

Der \begriff{Durchschnitt} zweier Mengen \(M\) und \(N\) (Kurzbezeichnung: \(M \cap N\)) ist die
Menge der Elemente, die sowohl in \(M\) als auch in \(N\) enthalten sind.
\[
	M \cap N = \{x \pipe x \in M \land x \in N\}
\]
\(M\) und \(N\) heißen disjunkt, wenn ihr Durchschnitt leer ist, das heißt wenn \(M \cap N = \emptyset\).
\begin{center}
	\begin{tikzpicture}
		\begin{scope}
			\clip (-1,0) circle (2);
			\fill[fhblue] (1,0) circle (2);
		\end{scope}

		\draw[black] (-1,0) circle (2) node[text=black, text width=2cm] {\(M\)};
		\draw[black] (1,0) circle (2) node[text=black, text width=2cm, align=right] {\(N\)};

		\node[text=white] at (0,0) {\(M \cap N\)};
	\end{tikzpicture}
\end{center}
\begin{samepage}
	Die \begriff{Vereinigung} zweier Mengen \(M\) und \(N\) (Kurzbezeichnung: \(M \cup N\)) ist die
	Menge der Elemente, die in \(M\) oder in \(N\) enthalten sind.
	\[
		M \cup N = \{x \pipe x \in M \lor x \in N\}
	\]
	\begin{center}
		\begin{tikzpicture}
			\fill[fhblue] (-1,0) circle (2);
			\fill[fhblue] (1,0) circle (2);

			\draw[black] (-1,0) circle (2) node[text=white, text width=2cm] {\(M\)};
			\draw[black] (1,0) circle (2) node[text=white, text width=2cm, align=right] {\(N\)};

			\node[text=white] at (0,0) {\(M \cup N\)};
		\end{tikzpicture}
	\end{center}
\end{samepage}
Die \begriff{Differenzmenge} zweier Mengen \(M\) und \(N\) (Kurzbezeichnung: \(M \backslash N\))
ist die Menge der Elemente die in M, aber nicht in N enthalten sind.
\[
	M \backslash N = \{x \pipe x \in M \land x \notin N\}
\]
\begin{center}
	\begin{tikzpicture}
		\begin{scope}
			\clip (-3, -2) rectangle (2, 2)
			(1,0) circle (2);
			\fill[fhblue] (-1,0) circle (2);
		\end{scope}

		\draw[black] (-1,0) circle (2) node[text=white, text width=3cm] {\(\stackon[5mm]{M \backslash N}{M}\)};
		\draw[black] (1,0) circle (2) node[text=black, text width=2cm, align=right] {\(N\)};
	\end{tikzpicture}
\end{center}
Ist im Umgang mit Mengen eine bestimmte Grundmenge \(G\) vereinbart (bei Zahlen zum Beispiel häufig \(\R\)),
so wird die Differenzmenge immer im Bezug auf diese Grundmenge gebildet, ohne dass sie explizit
erwähnt wird. Statt \(G \backslash N\) schreibt man dann \(\overline{N}\) und nennt \(\overline{N}\)
das \begriff{Komplement} von N.

\newpage
\begin{beispielenolist}
	\mbox{}\vspace*{-1cm}
	\[M=\{\triangle,\bigcirc,\square\}, N=\{\blacksquare, \bigcirc, \square\}\]
	\begin{alignat*}{3}
		M \cap N       & = \{\square, \bigcirc\}                          \\
		M \cup N       & = \{\triangle, \bigcirc, \square, \blacksquare\} \\
		M \backslash N & = \{\triangle\}
	\end{alignat*}
\end{beispielenolist}
\vspace*{-1cm}
\begin{samepage}
	\begin{bemerkung}
		Für die Mengenoperationen gelten Rechenregeln (vgl. Übung 1.13).
	\end{bemerkung}
\end{samepage}
\überschrift{Quantoren}

\begriff{Quantoren} stellen ein Bindeglied zwischen Aussagelogik und Mengenlehre dar.

An Stelle der Aussage
\[\textsf{Es gibt ein Element~} x \textsf{~in der Menge~} M \textsf{~mit der Eigenschaft~} E \textsf{.}\]
schreibt man kurz
\newcommand{\existsclause}[3]{\exists \; #1 \in #2 \hspace*{5mm} #3}
\[\existsclause{x}{M}{E}\]
Das Zeichen \(\exists\) heißt \begriff{Existenzquantor}.

An Stelle der Aussage
\[\textsf{Für alle~} x \textsf{~in der Menge~} M \textsf{~gilt die Eigenschaft~} E \textsf{.}\]
schreibt man kurz
\newcommand{\allclause}[3]{\forall \; #1 \in #2 \hspace*{5mm} #3}
\[\allclause{x}{M}{E}\]
Das Zeichen \(\forall\) heißt \begriff{Allquantor}.

\begin{beispielenolist}
	Die Aussage \textspaced{\(\existsclause{n}{\mathbb{N}}{n<0}\)} ist \falsch, denn natürliche Zahlen sind nicht negativ.

	Die Aussage \textspaced{\(\allclause{x}{\mathbb{Z}}{x \textsf{~ist durch 7 teilbar}}\)}
	ist \falsch, denn \(8 \in \mathbb{Z}\) und 8 ist nicht durch 7 teilbar.
\end{beispielenolist}

\überschrift{Unendliche Vereinigung, unendlicher Durchschnitt}

Sei \(I\) eine Menge, die wir als Menge der \begriff{Indizes} bezeichnen (jedes
Element von \(I\) ist ein \begriff{Index}). Für jedes \(i \in I\) sei eine
Menge \(M_i\) gegeben. Die Menge \(\bigcup\limits_{i \in I} M_i\), definiert durch
\[\bigcup\limits_{i \in I} M_i := \{x \pipe \existsclause{i}{I}{x \in M_i}\}\]
heißt \begriff{unendliche Vereinigung} der Mengen \(M_i\).

Die Menge \(\bigcap\limits_{i \in I} M_i\), definiert durch
\[\bigcap\limits_{i \in I} M_i := \{x \pipe \allclause{i}{I}{x \in M_i}\}\]
heißt \begriff{unendlicher Durchschnitt} der Mengen \(M_i\).

\begin{beispiele}
	\item \(I = \mathbb{N}\), \(M_i = \{x \in \mathbb{R} \pipe 0 < x < \frac{1}{i} \}\).
	\hspace{5mm} Dann ist \(\bigcap\limits_{i \in I} M_i = \emptyset\)
	\item \(I = \mathbb{N}\), \(M_i = \{-i, i\}\). \hspace{5mm} Dann ist
	\(\bigcup\limits_{i \in I} M_i = \mathbb{Z} \backslash \{0\}\)
	\item \(I = \{\triangle, \bigcirc, \square\}\), \(M_\triangle = \{3\}\),
	\(M_\bigcirc = \{0\}\), \(M_\square = \{4\}\). \vspace*{2mm} \newline
	\(\bigcup\limits_{i \in I} M_i = \{3, 0, 4\}\), \(\bigcap\limits_{i \in I} M_i = \emptyset\)
\end{beispiele}

\begin{definition}[Kartesisches Produkt]
	Sind \(M\) und \(N\) Mengen, so heißt die Menge \(M \times N\), definiert durch
	\[M \times N := \{(x, y) \pipe x \in M, y \in N\}\]
	also die Menge aller \begriff{geordneten Paare} \((x,y)\) mit
	\(x \in M\) und \(y \in N\), das \newline \begriff{kartesische Produkt} von \(M\) und \(N\).
\end{definition}

\begin{bemerkungenbsp}
	\item Geordnet heißt, dass etwa \((1,4)\) und \((4,1)\) verschiedene Elemente
	von \(\mathbb{N} \times \mathbb{N}\) sind.

	\item \begin{samepage}
		Das kartesische Produkt ist nicht kommutativ. \newline
		Beispiel: \(M=\{1,2\}\), \(N=\{a,b,c\}\). Dann gilt
		\begin{addmargin}[5mm]{0cm}
			\(M \times N = \{(1,a),(1,b),(1,c),(2,a),(2,b),(2,c)\}\), \newline
			\(N \times M = \{(a,1),(a,2),(b,1),(b,2),(c,1),(c,2)\}\), \newline
			das heißt \(M \times N \neq N \times M\). \newline
		\end{addmargin}
	\end{samepage}
	\vspace{-\baselineskip}

	\item Für \(k\) Mengen (\(k \in \mathbb{N}\), \(k \geq 2\)) \(M_1,M_2,\dots ,M_k\)
	kann man analog das k-fache kartesiche Produkt \(M_1 \times M_2 \times \dots
	\times M_k = \{(x_1, x_2, \dots , x_k) \pipe x_i \in M_i\}\) bilden. Die
	Elemente dieses Produktes heißen \begriff{geordnete \(k\)-Tupel}. Sind alle
	Mengen \(M_i\) gleich, \(M_1=M_2=\dots =M_k=M\), so schreibt man
	\(\underbrace{M \times M \times \dots \times M}_{k\textsf{~mal}} = M^k\).

	\item \begin{samepage} \(\mathbb{R}^2 = \mathbb{R} \times \mathbb{R}\) ist die
		Menge der kartesischen koordinaten in zwei Dimensionen. Die Elemente von
		\(\mathbb{R}^2\) können als Punkte im kartesischen Koordinatenssystem in
		der Ebene aufgefasst werden.
		\begin{center}
			\begin{tikzpicture}
				\begin{axis}[xmin=0, xmax = 5, ymin = 0, ymax = 3,
						axis lines = middle, enlargelimits = true,
						xtick distance=1, ytick distance=1,
						grid=major, grid style={dashed,gray!30},
						unit vector ratio={1 1}]
					\draw[dashed] (axis cs:0,2) -- (axis cs:4,2);
					\draw[dashed] (axis cs:4,0) -- (axis cs:4,2);
					\node at (axis cs:4,2) {\textbullet};
					\node[above] at (axis cs:4,2) {\((4,2)\)};
				\end{axis}
			\end{tikzpicture}
		\end{center}
	\end{samepage}
\end{bemerkungenbsp}

\subsection{Relationen}

\begin{definition}[Relation]
	Sind \(M\) und \(N\) Mengen, so heißt eine Teilmenge \(R \subset M \times N\) des
	kartesischen Produktes von \(M\) und \(N\) eine \begriff{Relation} auf \(M \times N\).

	Ist \(M = N\), so heißt \(R\) kurz eine Relation auf \(M\).

	Statt \((x,y) \in R\) schreibt man auch kurz \(x\,R\;y\).
\end{definition}

\begin{bemerkung}
	Relation heißt Beziehung. Eine Relation auf \(M \times N\) beschreibt nämlich eine
	Beziehung, die zwischen bestimmten Paaren von Elementen der Mengen \(M\) und \(N\)
	besteht. Als Teilmenge von \(M \times N\) enthält sie genau die Paare \((x,y)\), für
	die die Beziehung \(x\,R\;y\) gilt.
\end{bemerkung}

\begin{beispiel}
	Sei \(S\) die Menge der Studierenden des Fachbereichs ETI und \(F\) die Menge der
	angebotenen Fächer eines Studienganges. Dann ist die Menge
	\begin{gather*}
		B=\{(s,f) \pipe s \in S \textsf{~hat das Fach~} f \in F \textsf{~bestanden}\}\\
		B \subset S \times F
	\end{gather*}
	eine Relation auf \(S \times F\).
\end{beispiel}

\newcommand{\är}{Äquivalenzrelation}
\begin{definition}[reflexiv, symmetrisch, transitiv]
	Es sei \(R\) eine Relation auf der Menge \(M\). \(R\) heißt
	\begin{enumerate}[label=\alph*)]
		\item \begriff{reflexiv}, wenn für alle \(x \in M\) gilt \(x\,R\;x\),
		\item \begriff{symmetrisch}, wenn für alle \(x,y \in M\) gilt \(x\,R\;y \implies y\,R\;x\),
		\item \begriff{transitiv}, wenn für alle \(x,y,z \in M\) gilt \(x\,R\;y \land y\,R\;z \implies x\,R\;z\).
	\end{enumerate}

	Eine Relation heißt \begriff{\är}, wenn sie reflexiv, symmetrisch und transitiv ist.
\end{definition}

\begin{beispiele}
	\item Unter den Vergleichsoperatoren \(\leq, <, \geq, >, \neq, =\) für reelle Zahlen ist nur
	\(=\) eine \är ~auf \(\R\). Beispielsweise ist
	\begin{itemize}
		\item \(<\) nicht reflexiv, denn \(x < x\) ist nicht für alle \(x \in \R\) \wahr ~(nicht einmal
		      für irgendein \(x\))
		\item \(\geq\) nicht symmetrisch, denn \(x \geq y \implies y \geq x\) ist nicht für alle \(x,y \in \R\)
		      \wahr ~(z.B. nicht für \(x=17\), \(y=3\))
		\item \(\neq\) nicht transitiv, denn \(x \neq y \land y \neq z \implies x \neq z\) ist nicht
		      für alle \(x,y,z \in \R\) \wahr ~(z.B. nicht für \(x=3\), \(y=7\), \(z=3\))
	\end{itemize}

	\newpage
	\item \(\equiv_3 \;:= \{(b, b') \in \Z^2 \pipe b'-b \textsf{~ist durch 3 teilbar}\}\)
	ist eine \är. Beweis:
	\begin{enumerate}[label=\alph*)]
		\item Reflexivität: \\
		      \(b \equiv_3 b \equivalent 0\) ist durch 3 teilbar, und Letzteres ist für alle
		      \(b \in \Z\) \wahr. Also ist \(\equiv_3\) reflexiv.

		\item Symmetrie:
		      \begin{alignat*}{3}
			      b \equiv_3 b' & \implies b' - b \textsf{~ist durch 3 teilbar}                    \\
			                    & \implies b' - b = k \cdot 3 \textsf{~mit~} k \in \Z              \\
			                    & \implies b - b' = (-k) \cdot 3 \textsf{~mit~} k \in \Z           \\
			                    & \implies b - b' = l \cdot 3 \textsf{~mit~} l \in \Z \;\;(l = -k) \\
			                    & \implies b - b' \textsf{~ist durch 3 teilbar}                    \\
			                    & \implies b' \equiv_3 b
		      \end{alignat*}
		      Also ist \(\equiv_3\) symmetrisch.

		\item Transitivität:
		      \begin{alignat*}{3}
			      b \equiv_3 b' \land b' \equiv_3 b & \implies b' - b \textsf{~und~} b'' - b' \textsf{~sind durch 3 teilbar}                      \\
			                                        & \implies b' - b = k \cdot 3 \textsf{~und~} b'' - b' = k' \cdot 3 \textsf{~mit~} k,k' \in \Z \\
			                                        & \implies b'' - b = (k + k') \cdot 3                                                         \\
			                                        & \implies b'' - b = l \cdot 3 \textsf{~mit~} l \in \Z \;\;(l = k + k')                       \\
			                                        & \implies b'' - b \textsf{~ist durch 3 teilbar}                                              \\
			                                        & \implies b \equiv_3 b'' \textsf{~für alle~} b, b', b'' \in \Z
		      \end{alignat*}
		      Also ist \(\equiv_3\) transitiv.
	\end{enumerate}
	Da \(\equiv_3\) reflexiv, symmetrisch und transitiv ist, ist \(\equiv_3\) eine \är.

	\qed
\end{beispiele}

\newcommand{\äk}{Äquivalenzklasse}
\newcommand{\äkn}{Äquivalenzklassen}
Eine \är ~\(\sim\) auf einer Menge \(X\) zerlegt \(X\) in sogenannte \äkn.

\newpage
\begin{definition}[\äk]
	Ist \(\sim\) eine \är ~auf der Menge \(X\) und \(a \in X\), so heißt die Menge
	\[
		[a]_\sim := \{x \in X \pipe x \sim a\}
	\]
	die \begriff{\äk} von \(\sim\) mit dem \begriff{Repräsentanten} (oder \begriff{Vertreter}) \(a\).
\end{definition}

\begin{bemerkung}
	\([a]_\sim\) ist also die Menge aller Elemente von \(X\), die zu \(a\) in der Relation \(\sim\) stehen,
	darunter \(a\) selbst.
\end{bemerkung}

\begin{satz}
	Es sei \(X\) eine nichtleere Menge und \(\sim\) eine \är ~auf \(X\). Die Menge aller
	\äkn ~von \(\sim\) stellt eine \begriff{Partition} (oder \begriff{Zerlegung}) von \(X\) dar,
	das heißt
	\begin{enumerate}
		\item alle \äkn ~sind nichtleer,
		\item je zwei verschiedene \äkn ~sind disjunkt,
		\item die Vereinigung aller \äkn ~ist gleich \(X\).
	\end{enumerate}
\end{satz}

\begin{beispiel}[Restklassen]
	Zur \är ~\(\equiv_3\) gibt es genau drei \äkn:
	\begin{alignat*}{3}
		[0]_{\equiv_3} & = \{\dots,-6,-3,0,3,6,9,\dots\}  \\
		[1]_{\equiv_3} & = \{\dots,-5,-2,1,4,7,10,\dots\} \\
		[2]_{\equiv_3} & = \{\dots,-4,-1,2,5,8,11,\dots\}
	\end{alignat*}
	Sie sind nichtleer, paarweise disjunkt, und es gilt
	\[
		[0]_{\equiv_3} \cup [1]_{\equiv_3} \cup [2]_{\equiv_3} = \bigcup^2_{i=0} \; [i]_{\equiv_3} = \Z
	\]
	das heißt die \äkn ~\([i]_{\equiv_3}\) (\(i=0,1,2\)) bilden eine Partition von \(\Z\). Sie
	heißen \begriff{Restklassen modulo 3}.

	Allgemeiner bezeichnet man für \(n \in \N\) die \(n\) verschiedenen
	\äkn ~\([i]_{\equiv_n}\) (\(i=0,\dots,n-1\)) der \är
	\[
		\equiv_n := \{(b,b') \in \Z \pipe (b' - b) \textsf{~ist durch~} n \textsf{~teilbar}\}
	\]
	als die \begriff{Restklassen modulo n}.
\end{beispiel}

\subsection{Abbildungen}

\newcommand{\abb}{Abbildung}
\newcommand{\abbn}{Abbildungen}
\newcommand{\nach}{\rightarrow}
\begin{definition}[\abb]
	Gegeben seien zwei Mengen \(M\) und \(N\) und eine Zuordnungsvorschrift, die jedem\\
	\(x \in M\) genau ein \(y \in N\) zuordnet. Dann ist durch \(M,N\) und diese
	Zuordnungsvorschrift eine \begriff{\abb} ~\(f\) gegeben.

	Man schreibt
	\begin{alignat*}{3}
		               & f: M \nach N      &  & \textsf{~mit~} x \mapsto f(x)              \\
		\textsf{oder~} & f: x \mapsto f(x) &  & \textsf{~mit~} x \in M, f(x) \in N         \\
		\textsf{oder~} & y = f(x)          &  & \textsf{~mit~} x \in M, y \in N \textsf{.}
	\end{alignat*}
\end{definition}

\überschrift{Sprechweisen und Notationen}

Man nennt \(f\) die durch \(y = f(x)\) definierte Abbildung von \(M\) nach \(N\). Vor allem wenn
\(M\) und \(N\) Teilmengen von \(\R\) sind, nennt man \abbn	auch Funktionen. Weiter nennt man

\begin{tabular}{p{0.3\textwidth}>{\arraybackslash}p{0.6\textwidth}}
	\(x\)                          & \begriff{Argument}, \begriff{Variable} von \(f\)                                                                                                          \\
	\(y\), \(f(x)\)                & \begriff{Wert} oder \begriff{Funktionswert} von \(f\) an der Stelle \(x\) oder bei \(x\), \begriff{Bild} von \(x\) (\(x\) ist \begriff{Urblid} von \(y\)) \\
	\(x \mapsto f(x)\), \(y=f(x)\) & \begriff{Zuordnungsvorschrift} für \(f\)                                                                                                                  \\
	\(M\),\(D_f\)                  & \begriff{Definitionsmenge} von \(f\)                                                                                                                      \\
	\(N\)                          & \begriff{Zielmenge} von \(f\)
\end{tabular}

Zwei \abbn ~\(f: M \nach N\) und \(g: U \nach V\) heißen gleich, wenn
\(M=U\), \(N=V\) und \(f(x)=g(x)\) für alle \(x \in M\). Man schreibt dann \(f = g\).

Die Bezeichnung der Variablen in der Zuordnungsvorschrift, das heißt die Wahl der Buchstaben
ist beliebig. So bedeuten folgende Zuordnungsvorschriften alle das gleiche:
\[
	x \mapsto x^2 \qquad v \mapsto v^2 \qquad y \mapsto y^2
\]
Die Menge aller Funktionswerte \(f(x)\) mit \(x \in M\) heißt \begriff{Bildmenge} oder
\begriff{Wertemenge} von \(f\) und wird mit \(f(M)\) oder \(W_f\) bezeichnet, also
\[
	f(M) = \{y \in N \pipe \existsclause{x}{M}{f(x)=y}\}
\]

Ist \(U \subset M\) eine Teilmenge der Definitionsmenge, so heißt die Menge
aller Funktionswerte \(f(x)\) mit \(x \in U\) das \begriff{Bild} von \(U\)
und wird mit \(f(U)\) bezeichnet, also
\[
	f(U) = \{y \in N \pipe \existsclause{x}{U}{f(x)=y}\}
\]

Ist \(V \subset N\) eine Teilmenge der Zielmenge, so heißt die Menge aller
Argumente \(x \in M\) mit \(f(x) \in V\) das \begriff{Urbild} von \(V\) und
wird mit \(f^{-1}(V)\) bezeichnet, also
\[
	f^{-1}(V) = \{x \in M \pipe f(x) \in V\}
\]

\begin{beispiele}
	\item Die Zuordnungsvorschrift \(x \mapsto y\) mit \(y^2 = x\) definiert keine Abbildung
	von \(\R\) nach \(\R\), weil zum Beispiel \(x = -1\) kein \(y \in \R\) zugeordnet ist,
	und weil \(x = 4\) mit \(y = 2\) und \(y = -2\) zwei \(y \in \R\) zugeordnet sind.

	Ebenso definiert die Zuordnungsvorschrift \(x \mapsto \frac{x+1}{x}\) keine \abb ~von
	\(\R\) nach \(\R\), weil \(x = 0\) kein Funktionswert zugeordnet ist. Durch
	\(f: \N \nach \R\) mit \(x \mapsto \frac{x+1}{x}\) ist hingegen eine \abb	~von
	\(\N\) nach \(\R\) definiert.

	\newpage
	\item Die folgende Grafik beschreibt eine \abb ~\(f\):
	\begin{center}
		\begin{tikzpicture}[-latex]
			\draw (-3,0) circle (2);

			\node[left] at (-5,0) {\(M\)};
			\node at (-4,1) {\textbullet};
			\node[below left] at (-4,1) {\(a\)};
			\node at (-4,-1) {\textbullet};
			\node[above left] at (-4,-1) {\(c\)};
			\node at (-2,1) {\textbullet};
			\node[below left] at (-2,1) {\(b\)};
			\node at (-2,-1) {\textbullet};
			\node[above left] at (-2,-1) {\(d\)};

			\draw (3,0) circle (2);

			\node[right] at (5,0) {\(N\)};
			\node at (2,1) {\textbullet};
			\node[below right] at (2,1) {\(1\)};
			\node at (4,0) {\textbullet};
			\node[right] at (4,0) {\(2\)};
			\node at (2,-1) {\textbullet};
			\node[above right] at (2,-1) {\(3\)};

			\node[above] at (0,2) {\(f\)};

			\path (-4,1) edge[bend left] (1.9,1.1);
			\path (-2,1) edge[bend right] (1.9,0.9);
			\path (-4,-1) edge[bend right] (1.9,-1.1);
			\path (-2,-1) edge[bend left] (1.9,-0.9);
		\end{tikzpicture}
	\end{center}
	\abb ~\(f\) mit \(M=\{a,b,c,d\}\), \(N=\{1,2,3\}\), \(f(M)=\{1,3\} \neq N\).

	\(1\) ist Bild von \(a\) und \(b\). \(3\) ist Bild von \(c\) und \(d\). \(a\) und \(b\)
	sind Urbilder von \(1\). \(c\) und \(d\) sind Urbilder von \(3\). \(2\) hat kein Urbild.

	Das Bild von \(U=\{a,b\}\) ist \(f(U)=\{1\}\).
	Das Urbild von \(V=\{2\}\) ist \(f^{-1}(V)=\emptyset\).
	Das Urbild von \(V'=\{2,3\}\) ist \(f^{-1}(V')=\{c,d\}\).

	Dagegen ist
	\begin{center}
		\begin{tikzpicture}[-latex]
			\draw (-3,0) circle (2);

			\node[left] at (-5,0) {\(M\)};
			\node at (-4,1) {\textbullet};
			\node[below left] at (-4,1) {\(a\)};
			\node at (-4,-1) {\textbullet};
			\node[above left] at (-4,-1) {\(c\)};
			\node at (-2,1) {\textbullet};
			\node[below left] at (-2,1) {\(b\)};
			\node at (-2,-1) {\textbullet};
			\node[above left] at (-2,-1) {\(d\)};

			\draw (3,0) circle (2);

			\node[right] at (5,0) {\(N\)};
			\node at (2,1) {\textbullet};
			\node[below right] at (2,1) {\(1\)};
			\node at (4,0) {\textbullet};
			\node[right] at (4,0) {\(2\)};
			\node at (2,-1) {\textbullet};
			\node[above right] at (2,-1) {\(3\)};

			\node[above] at (0,2) {\(f\)};

			\path (-4,1) edge[bend left] (1.9,1.1);
			\path (-2,1) edge[bend right] (1.9,0.9);
			\path (-4,-1) edge[bend left] (1.9,-0.9);
			\path (-4,-1) edge[bend right] (3.9,-0.1);
		\end{tikzpicture}
	\end{center}
	keine \abb:
	\begin{itemize}
		\item \(d\) hat kein Bild
		\item \(c\) hat zwei Bilder
	\end{itemize}

	\item Durch die Zuordnung \(f:\{\textsf{a},\textsf{b},\textsf{c},\dots,\textsf{z}\} \nach \N\) mit
	\begin{center}
		\begin{tabular}{c || c | c | c | c | c}
			\(x\)    & a      & b      & c      & \dots & z     \\
			\hline
			\(f(x)\) & \(26\) & \(25\) & \(24\) & \dots & \(1\)
		\end{tabular}
	\end{center}
	ist eine \abb ~definiert.
\end{beispiele}

\newpage
\begin{definition}[injektiv, surjektiv, bijektiv]
	Es sei \(f: M \nach N\) eine \abb. \(f\) heißt
	\begin{enumerate}[label=\alph*)]
		\item \begriff{injektiv}, wenn für alle \(x_1,x_2\in M\) gilt:
		      \[
			      x_1 \neq x_2 \implies f(x_1) \neq f(x_2)
		      \]
		\item \begriff{surjektiv}, wenn für jedes \(y \in N\) ein \(x \in M\)
		      existiert mit \(f(x) = y\):
		      \[
			      \allclause{y}{N}{\existsclause{x}{M}{f(x) = y}}
		      \]
		\item \begriff{bijektiv}, wenn \(f\) injektiv und surjektiv ist.
	\end{enumerate}
\end{definition}

\begin{bemerkungennolist}
	Bei einer injektiven \abb ~werden verschiedene Elemente der Definitionsmenge
	stets auf verschiedene Elemente der Zielmenge abgebildet.

	Bei einer surjektiven \abb ~hat jedes Element der Zielmenge ein Urbild, und
	damit ist die Zielmenge gleich der Wertemenge.
\end{bemerkungennolist}

\begin{beispiele}
	\item Die folgenden \abbn ~sind:
	\begin{center}\setlength{\tabcolsep}{5mm}
		\begin{tabular}{c c}
			\begin{tikzpicture}[-latex, scale=0.5]
				\draw (-3,0) circle (2);

				\node[left] at (-5,0) {\(M\)};
				\node at (-3,1) {\textbullet};
				\node at (-3,-1) {\textbullet};

				\draw (3,0) circle (2);

				\node[right] at (5,0) {\(N\)};
				\node at (3,1) {\textbullet};
				\node at (3,-1) {\textbullet};

				\node at (0,0) {\(f\)};

				\path (-3,1) edge[bend left] (2.9,1.1);
				\path (-3,-1) edge[bend right] (2.9,0.9);

				\node[below] at (0,-2) {nicht injektiv, nicht surjektiv};
			\end{tikzpicture}
			 &
			\begin{tikzpicture}[-latex, scale=0.5]
				\draw (-3,0) circle (2);

				\node[left] at (-5,0) {\(M\)};
				\node at (-3,1) {\textbullet};
				\node at (-3,-1) {\textbullet};

				\draw (3,0) circle (2);

				\node[right] at (5,0) {\(N\)};
				\node at (3,0) {\textbullet};

				\node at (0,0) {\(f\)};

				\path (-3,1) edge[bend left] (2.9,0.1);
				\path (-3,-1) edge[bend right] (2.9,-0.1);

				\node[below] at (0,-2) {nicht injektiv, surjektiv};
			\end{tikzpicture}
			\\
			\begin{tikzpicture}[-latex, scale=0.5]
				\draw (-3,0) circle (2);

				\node[left] at (-5,0) {\(M\)};
				\node at (-3,1) {\textbullet};
				\node at (-3,-1) {\textbullet};

				\draw (3,0) circle (2);

				\node[right] at (5,0) {\(N\)};
				\node at (2.5,1) {\textbullet};
				\node at (2.5,-1) {\textbullet};
				\node at (4,0) {\textbullet};

				\node at (0,0) {\(f\)};

				\path (-3,1) edge[bend left] (2.4,1.1);
				\path (-3,-1) edge[bend right] (2.4,-1.1);

				\node[below] at (0,-2) {injektiv, nicht surjektiv};
			\end{tikzpicture}
			 &
			\begin{tikzpicture}[-latex, scale=0.5]
				\draw (-3,0) circle (2);

				\node[left] at (-5,0) {\(M\)};
				\node at (-3,1) {\textbullet};
				\node at (-3,-1) {\textbullet};

				\draw (3,0) circle (2);

				\node[right] at (5,0) {\(N\)};
				\node at (3,1) {\textbullet};
				\node at (3,-1) {\textbullet};

				\node at (0,0) {\(f\)};

				\path (-3,1) edge[bend left] (2.9,1.1);
				\path (-3,-1) edge[bend right] (2.9,-1.1);

				\node[below] at (0,-2) {injektiv, surjektiv};
			\end{tikzpicture}
		\end{tabular}
	\end{center}

	\item Die \abb ~\(g: \N^2 \nach \N\), \((m,n) \mapsto \textrm{ggt}(m,n)\) ordnet
	jedem Paar zweier natürlicher Zahlen ihren größten gemeinsamen Teiler zu.

	Die \abb ~ist surjektiv, denn \(\textrm{ggt}(n,n)=n\).

	Sie ist nicht injektiv, denn \(\textrm{ggt}\underbrace{(3,6)}_{\in \N^2}=\textrm{ggt}\underbrace{(9,12)}_{\in \N^2}\)
	\begin{align*}
		(3,6) \neq (9,12) & \stackon{\implies}{\text{\Lightning}} \textrm{ggt}(3,6) \neq \textrm{ggt}(9,12) \\
		                  & \stackon{\implies}{\text{\Lightning}} 3 \neq 3
	\end{align*}
\end{beispiele}

\begin{definition}[Umkehrabbildung]
	Ist \(f: M \nach N\), \(x \mapsto f(x)\) eine bijektive \abb, so wird
	durch \(g: N \nach M\), \(y \mapsto x\) mit \(y = f(x)\) eine \abb ~definiert.
	\(g\) heißt \begriff{Umkehrabbildung} oder \begriff{inverse \abb} zu \(f\) und
	man sagt, \(f\) sei umkehrbar. Die Umkehrabbildung wird mit \(f^{-1}\) bezeichnet.
\end{definition}

\end{document}